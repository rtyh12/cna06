\documentclass {article}
\usepackage{amsmath}
\setlength{\parindent}{0cm}

\usepackage{hyperref}
\usepackage{graphicx}
\usepackage{subcaption}
\usepackage[margin=1.5in]{geometry}
\usepackage{tikz}			% for graphs
\usepackage{float}			% for having the graphs not fly off into space
\usetikzlibrary{arrows}

\usepackage{fancyhdr}
\pagestyle{fancy}
\lhead{\textbf{Complex Network Analysis} \\ Assignment 6\\}
\rhead{Maria Kagkeli \\ Maria Regina Lily \\ Mihai Verzan}
\headheight 10pc
\voffset -10pc

\begin{document}



% problem 1
\section{Degree Correlation Coefficient}
For all the computations, please see \texttt{Problem 6-1.ipynb}
\subsection{Degree Correlation Matrix}
\[
 	E = 
 	\begin{bmatrix}
 	0     &  0   & 1/8 \\
 	0     & 1/4 & 1/4 \\
  	1/8  & 1/4 & 0
 	\end{bmatrix}
\]

\subsection{Probabilities $q_k$}
\[{
\begin{array}{cccccccc}
q_1 &=& 0&+&0&+&1/8 =&1/8 \\
q_2 &=& 0&+&1/4&+&1/4 =&1/2 \\
q_3 &=& 1/8&+&1/4&+&0 =&3/8 
\end{array}
}\]
\subsection{Degree Correlation Coefficient}
$r = -0.7142857142857143$\\

Based on r, the given network is disassortative ($r < 0$)
\newpage



%problem 2
\section{Degree Correlations in Random Graphs}
\subsection{}

\subsection{}

\subsection{}
\begin{align*}
	\frac{P \left( a_{ij} = 1 | a_{xy} = 0 \right)}{P \left( a_{ij}=0 \right)} &=\frac{L}{\frac{N(N-1)}{2}-1} \cdot \frac{\frac{N(N-1)}{2}}{L}\\
	&= \frac{2}{N(N-1)-2} \cdot \frac{N(N-1)}{2}\\
	&= \frac{N(N-1)}{N(N-1)-2}\\
	&&\lim\limits_{N \to \infty} \frac{N(N-1)}{N(N-1)-2} = 1
\end{align*}

\begin{align*}
	\frac{P \left( a_{ij} = 1 | a_{xy} = 1 \right)}{P \left( a_{ij}=0 \right)} &=\frac{L-1}{\frac{N(N-1)}{2}-1} \cdot \frac{\frac{N(N-1)}{2}}{L}\\
	&=\frac{L-1}{\frac{N(N-1)-2}{2}} \cdot \frac{N(N-1)}{2L}\\
	&= \frac{N(L-1)(N-1)}{\frac{2LN(N-1)-2}{2}}\\
	&= \frac{N(L-1)(N-1)}{LN(N-1)-2}\\
	&&\lim\limits_{N \to \infty}  \frac{N(L-1)(N-1)}{LN(N-1)-2} = \frac{L-1}{L}
\end{align*}

\subsection{}
for $G(N,p)$ model, $r_0' = r_1' = 1$, in other words, the conditional probabilities don't change and is always the same as $p(a_{ij} = 1)$.
This is the because the probability p that an edge exist  in a $G(N,p)$ model is fixed, unlike in an $G(N,L)$ model

\subsection{}
\newpage




%problem 3
\section{Degree Correlations and Assortativity}
\end{document}